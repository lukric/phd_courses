\documentclass{beamer}\usepackage[]{graphicx}\usepackage[]{color}
%% maxwidth is the original width if it is less than linewidth
%% otherwise use linewidth (to make sure the graphics do not exceed the margin)
\makeatletter
\def\maxwidth{ %
  \ifdim\Gin@nat@width>\linewidth
    \linewidth
  \else
    \Gin@nat@width
  \fi
}
\makeatother

\definecolor{fgcolor}{rgb}{0.345, 0.345, 0.345}
\newcommand{\hlnum}[1]{\textcolor[rgb]{0.686,0.059,0.569}{#1}}%
\newcommand{\hlstr}[1]{\textcolor[rgb]{0.192,0.494,0.8}{#1}}%
\newcommand{\hlcom}[1]{\textcolor[rgb]{0.678,0.584,0.686}{\textit{#1}}}%
\newcommand{\hlopt}[1]{\textcolor[rgb]{0,0,0}{#1}}%
\newcommand{\hlstd}[1]{\textcolor[rgb]{0.345,0.345,0.345}{#1}}%
\newcommand{\hlkwa}[1]{\textcolor[rgb]{0.161,0.373,0.58}{\textbf{#1}}}%
\newcommand{\hlkwb}[1]{\textcolor[rgb]{0.69,0.353,0.396}{#1}}%
\newcommand{\hlkwc}[1]{\textcolor[rgb]{0.333,0.667,0.333}{#1}}%
\newcommand{\hlkwd}[1]{\textcolor[rgb]{0.737,0.353,0.396}{\textbf{#1}}}%
\let\hlipl\hlkwb

\usepackage{framed}
\makeatletter
\newenvironment{kframe}{%
 \def\at@end@of@kframe{}%
 \ifinner\ifhmode%
  \def\at@end@of@kframe{\end{minipage}}%
  \begin{minipage}{\columnwidth}%
 \fi\fi%
 \def\FrameCommand##1{\hskip\@totalleftmargin \hskip-\fboxsep
 \colorbox{shadecolor}{##1}\hskip-\fboxsep
     % There is no \\@totalrightmargin, so:
     \hskip-\linewidth \hskip-\@totalleftmargin \hskip\columnwidth}%
 \MakeFramed {\advance\hsize-\width
   \@totalleftmargin\z@ \linewidth\hsize
   \@setminipage}}%
 {\par\unskip\endMakeFramed%
 \at@end@of@kframe}
\makeatother

\definecolor{shadecolor}{rgb}{.97, .97, .97}
\definecolor{messagecolor}{rgb}{0, 0, 0}
\definecolor{warningcolor}{rgb}{1, 0, 1}
\definecolor{errorcolor}{rgb}{1, 0, 0}
\newenvironment{knitrout}{}{} % an empty environment to be redefined in TeX

\usepackage{alltt}
%\setbeamertemplate{background}{\includegraphics[width=\paperwidth,height=\paperheight,keepaspectratio]{ages_background2.png}}

\usepackage[ngerman,english]{babel}
\usepackage[T1]{fontenc}
\usepackage[utf8]{inputenc}

\usetheme{Boadilla}

\definecolor{ages}{RGB}{204,164,0}
\usecolortheme[named=ages]{structure}

\setbeamercovered{transparent}
\beamertemplatenavigationsymbolsempty
\setbeamertemplate{footline}[frame number]

\let\otp\titlepage
\renewcommand{\titlepage}{\otp\addtocounter{framenumber}{-1}}
\IfFileExists{upquote.sty}{\usepackage{upquote}}{}
\begin{document}
%%%%
\title{PITA -- Pertussis Infection and Transmission in Austria}
\subtitle{FFG Bridge project proposal\\
%homework\\
SE Fundamental and Applied Research, January 2019}
\author{Lukas Richter}
\institute{AGES, Austrian Agency for Health and Food Safety}
\date[]{23.01.2019} %{22.01.2019}

% Title slide
%%%%%%%%%%%%%%%%%%%%%%%%%%%%%%%%%%%%%%%%%%%%%%
% \frame{
%   \titlepage
% }
\begin{frame}[plain]
 \titlepage
\end{frame}
%%%%%%%%%%%%%%%%%%%%%%%%%%%%%%%%%%%%%%%%%%%%%%

% slide 1
%%%%%%%%%%%%%%%%%%%%%%%%%%%%%%%%%%%%%%%%%%%%%%
\begin{frame}[fragile]{\scriptsize{PITA -- Pertussis Infection and Transmission in Austria}}
\begin{center}

\begin{block}{Background}
\begin{itemize}
  \item Pertussis infection resurging in AT
  \item Cases increase in all age groups
  \item Vaccine coverage high in children
\end{itemize}
\end{block}

\begin{block}{Objective and Methods}
Use Whole Genome Sequencing (WGS) together with mathematical modelling to evaluate the current vaccine efficacy and the current vaccination strategy against \textit{B. pertussis} $\rightarrow$ study and formulate interventions
\end{block}

\begin{block}{Expected Results}
\begin{itemize}
  \item Vaccination strategy ineffective (no booster used by population)
  \item Vaccine does not cover circulating \textit{B. pertussis} strains
\end{itemize}
\end{block}


\end{center}
\end{frame}
%%%%%%%%%%%%%%%%%%%%%%%%%%%%%%%%%%%%%%%%%%%%%%

% slide 2
%%%%%%%%%%%%%%%%%%%%%%%%%%%%%%%%%%%%%%%%%%%%%%
\begin{frame}[fragile]{\scriptsize{PITA -- Pertussis Infection and Transmission in Austria}}
\begin{center}

\begin{block}{Participating Institutions}
\begin{itemize}
  \item AGES, Austrian Agency for Health and Food Safety (scientific)
  \item BASGK, Federal Ministry of Labour, Social Affairs, Health and Consumer Protection (stakeholder, vaccine programme)
  \item GSK, GlaxoSmithKline plc (stakeholder, vaccine producer)
\end{itemize}
\end{block}

\begin{block}{Expected Costs and Duration}
\begin{itemize}
  \item 18 months, start in July 2019
  \item 2 PhD students (75\% time; 1 microbiologist, 1 mathematician)
  \item Total cost: 166,901.59 EUR 
  \item Expected FFG funding quote: 60\%
\end{itemize}
\end{block}

\end{center}
\end{frame}
%%%%%%%%%%%%%%%%%%%%%%%%%%%%%%%%%%%%%%%%%%%%%%

% % slide 3
% %%%%%%%%%%%%%%%%%%%%%%%%%%%%%%%%%%%%%%%%%%%%%%
% \begin{frame}[fragile]{Background}
% \begin{center}
% \begin{itemize}
%   \item 1
%   \item 2
% \end{itemize}
% \end{center}
% \end{frame}
% %%%%%%%%%%%%%%%%%%%%%%%%%%%%%%%%%%%%%%%%%%%%%%

\end{document}


