\documentclass{beamer}\usepackage[]{graphicx}\usepackage[]{color}
%% maxwidth is the original width if it is less than linewidth
%% otherwise use linewidth (to make sure the graphics do not exceed the margin)
\makeatletter
\def\maxwidth{ %
  \ifdim\Gin@nat@width>\linewidth
    \linewidth
  \else
    \Gin@nat@width
  \fi
}
\makeatother

\definecolor{fgcolor}{rgb}{0.345, 0.345, 0.345}
\newcommand{\hlnum}[1]{\textcolor[rgb]{0.686,0.059,0.569}{#1}}%
\newcommand{\hlstr}[1]{\textcolor[rgb]{0.192,0.494,0.8}{#1}}%
\newcommand{\hlcom}[1]{\textcolor[rgb]{0.678,0.584,0.686}{\textit{#1}}}%
\newcommand{\hlopt}[1]{\textcolor[rgb]{0,0,0}{#1}}%
\newcommand{\hlstd}[1]{\textcolor[rgb]{0.345,0.345,0.345}{#1}}%
\newcommand{\hlkwa}[1]{\textcolor[rgb]{0.161,0.373,0.58}{\textbf{#1}}}%
\newcommand{\hlkwb}[1]{\textcolor[rgb]{0.69,0.353,0.396}{#1}}%
\newcommand{\hlkwc}[1]{\textcolor[rgb]{0.333,0.667,0.333}{#1}}%
\newcommand{\hlkwd}[1]{\textcolor[rgb]{0.737,0.353,0.396}{\textbf{#1}}}%

\usepackage{framed}
\makeatletter
\newenvironment{kframe}{%
 \def\at@end@of@kframe{}%
 \ifinner\ifhmode%
  \def\at@end@of@kframe{\end{minipage}}%
  \begin{minipage}{\columnwidth}%
 \fi\fi%
 \def\FrameCommand##1{\hskip\@totalleftmargin \hskip-\fboxsep
 \colorbox{shadecolor}{##1}\hskip-\fboxsep
     % There is no \\@totalrightmargin, so:
     \hskip-\linewidth \hskip-\@totalleftmargin \hskip\columnwidth}%
 \MakeFramed {\advance\hsize-\width
   \@totalleftmargin\z@ \linewidth\hsize
   \@setminipage}}%
 {\par\unskip\endMakeFramed%
 \at@end@of@kframe}
\makeatother

\definecolor{shadecolor}{rgb}{.97, .97, .97}
\definecolor{messagecolor}{rgb}{0, 0, 0}
\definecolor{warningcolor}{rgb}{1, 0, 1}
\definecolor{errorcolor}{rgb}{1, 0, 0}
\newenvironment{knitrout}{}{} % an empty environment to be redefined in TeX

\usepackage{alltt}

%\setbeamertemplate{background}{\includegraphics[width=\paperwidth,height=\paperheight,keepaspectratio]{ages_background2.png}}

\usepackage[ngerman,english]{babel}
\usepackage[T1]{fontenc}
\usepackage[utf8]{inputenc}
\usepackage{comment}
\usepackage{colortbl}

\usepackage[backend = bibtex, natbib = true, style = numeric, sorting = none]{biblatex}
\addbibresource{../../references/mybib.bib}

\usetheme{Boadilla}

\definecolor{ages}{RGB}{204,164,0}
\usecolortheme[named=ages]{structure}

\setbeamercovered{transparent}
\beamertemplatenavigationsymbolsempty
\setbeamertemplate{footline}[frame number]

\newcommand{\Sin}[1]{\sin\left(#1\right)}%
\newcommand{\Cos}[1]{\cos\left(#1\right)}%
\newcommand{\Log}[1]{\log\left(#1\right)}%
\IfFileExists{upquote.sty}{\usepackage{upquote}}{}
\begin{document}
\graphicspath{{../../chapters/figures/IPD/}, {../../chapters/figures/}}
%\graphicspath{{D:/diss/chapters/figures/IPD/}} %{{paste0(getwd(), "/chapters/figures/IPD/")}} %{../figures/IPD/}
%% initialize %%









%%%%
\title{Statistical modelling in public health: segmented regression and methods to model low count data}
\subtitle{Dr. School seminar, Summer 2017}
\author{Lukas Richter}
\date[]{02.06.2017}

\titlegraphic{
  \includegraphics[width=2.0cm]{logo_ages}\hspace*{3cm}~%
  \includegraphics[width=2.0cm]{logo_tugraz}
}


%%%%%%%%%%%%%%%%%%%%%%%%%%%%%%%%%%%%%%%%%%%%%%
\frame{
  \titlepage
}
%%%%%%%%%%%%%%%%%%%%%%%%%%%%%%%%%%%%%%%%%%%%%%


%%%%%%%%%%%%%%%%%%%%%%%%%%%%%%%%%%%%%%%%%%%%%%
\begin{frame}[fragile]{Background}
\begin{center}
\begin{itemize}
  \item Streptococcus pneumoniae-related infections:\\
  among children main cause of
  \begin{itemize}
    \item meningitis
    \item bacterial pneumonia
    \item sepsis
  \end{itemize}
  \item 90 distinct pneumococcal serotypes
  \item Only a small number account for invasive pneumococcal disease (IPD)
  \item January 2012: pneumococcal conjugate vaccine introduced in the national childhood immunisation program:
  \begin{itemize}
    \item covering 10 serotypes (PCV10)
    \item administered at 3rd, 5th and 12th month of life
    \item funded
  \end{itemize}  
  \item Other vaccines: PCV13, PPV23
\end{itemize}
\end{center}
\end{frame}
%%%%%%%%%%%%%%%%%%%%%%%%%%%%%%%%%%%%%%%%%%%%%%

%%%%%%%%%%%%%%%%%%%%%%%%%%%%%%%%%%%%%%%%%%%%%%
\begin{frame}[fragile]{Objective}
\begin{block}{Objective}
  Our aim was to evaluate direct and indirect longitudinal effects of introducing a 10-valent pneumococcal conjugate vaccine in the Austrian childhood immunization program in 2012 on the occurrence of invasive pneumococcal disease in the Austrian population.
\end{block}
\begin{center}
\begin{itemize}
  \item Analyse case-based data of confirmed IPD from 2009--2016/05
  \item Use information on age, date of diagnosis and serotype
\end{itemize}
\end{center}
\end{frame}
%%%%%%%%%%%%%%%%%%%%%%%%%%%%%%%%%%%%%%%%%%%%%%


%%%%%%%%%%%%%%%%%%%%%%%%%%%%%%%%%%%%%%%%%%%%%%
\begin{frame}[fragile]{Datasources}
Merged data from 4 sources:
\begin{center}
\begin{itemize}
  \item National surveillance data on IPD, AGES Vienna
  \item National reference laboratory for IPD, AGES, Graz
  \item Surveillance-study on IPD among <5 yeras old operated by the Institute of Specific Prophylaxis and Tropical Medicine at the Medical University Vienna% and funded by a pharmaceutical company
  \item Population data from Statistics Austria
\end{itemize}
\end{center}
\end{frame}
%%%%%%%%%%%%%%%%%%%%%%%%%%%%%%%%%%%%%%%%%%%%%%

%%%%%%%%%%%%%%%%%%%%%%%%%%%%%%%%%%%%%%%%%%%%%%
\begin{frame}[fragile]{Method}
\begin{center}
\begin{itemize}
  \item Transform data into time series
  \item Interpret intervention as interruption of the time series\\
  $\Rightarrow$ interrupted time series
  \item Segmented regression allows detection of changes when comparing adjacent segments regarding
  \begin{itemize}
    \item the secular trend
    \item one-time (sudden) changes
    \item seasonal pattern
  \end{itemize}
  \item Apply a segmented negative binomial regression model
  \begin{itemize}
    \item monthly IPD incidence data
    \item period: January 2009 -- May 2016
    \item non-vaccine target groups, 5--49 and $\ge$50 years old
    \item serotype subgroups (e.g. PCV10 serotypes)
  \end{itemize}
\end{itemize}
\end{center}
\end{frame}
%%%%%%%%%%%%%%%%%%%%%%%%%%%%%%%%%%%%%%%%%%%%%%

%%%%%%%%%%%%%%%%%%%%%%%%%%%%%%%%%%%%%%%%%%%%%%
\begin{frame}[fragile]{Study periods}
\begin{center}
\begin{figure}
  \centering
  \caption{Monthly incidence rate of total IPD per 100,000 population including model segments (arrows), Austria, 2009 -- 2016/05.}
  \includegraphics[width=0.7\linewidth]{diss_segments.png} %[width=0.8\linewidth]
  \label{fig:IPDsegments}
\end{figure}
\end{center}
\end{frame}
%%%%%%%%%%%%%%%%%%%%%%%%%%%%%%%%%%%%%%%%%%%%%%

%%%%%%%%%%%%%%%%%%%%%%%%%%%%%%%%%%%%%%%%%%%%%%
\begin{frame}[fragile]{Definition of outcome}
\begin{table}[]
\centering
\caption{Definition of the intervention and control outcome.}
\begin{tabular}{lll}
\hline
Age-group  & Type of Outcome & Outcome definition \\\hline
<5         & Intervention    & PCV10 ST-IPD       \\
           & Control         & Non PCV10 ST-IPD   \\
           & Control         & 3, 6A, 19A ST-IPD  \\\hline
$\ge$50     & Intervention    & PCV10 ST-IPD       \\
           & Control         & Non PCV10 ST-IPD   \\
           & Control         & PPV23 ST-IPD       \\
           & Control         & Non PPV23 ST-IPD   \\
           & Control         & 3, 6A, 19A ST-IPD  \\\hline
5--49      & Control         & PCV10 ST-IPD       \\
           & Control         & Non PCV10 ST-IPD   \\\hline
\end{tabular}
\end{table}
\end{frame}
%%%%%%%%%%%%%%%%%%%%%%%%%%%%%%%%%%%%%%%%%%%%%%

%%%%%%%%%%%%%%%%%%%%%%%%%%%%%%%%%%%%%%%%%%%%%%
\begin{frame}[fragile]{Model}
\begin{center}
\begin{align*}
\begin{split}
\Log{Y_t}\, =&\, \Log{\text{pop}_t}+\beta_0+\beta_1\,\text{time}_t+\beta_2\,\Sin{\frac{2\,\pi\,t}{12}}+\beta_3\,\Cos{\frac{2\,\pi\,t}{12}} \\
  +& \beta_4\,\text{vaccine}_t +\beta_5\,\text{time\_after\_vaccine}_t + \beta_6\,\sin_{\text{aftervaccine}} \left(\frac{2\,\pi\,t}{12}\right)\\
  +&\,\beta_7\,\cos_{\text{aftervaccine}}\left(\frac{2\,\pi\,t}{12}\right)+\varepsilon_t.
\end{split}
\end{align*}

\end{center}
\end{frame}
%%%%%%%%%%%%%%%%%%%%%%%%%%%%%%%%%%%%%%%%%%%%%%

%%%%%%%%%%%%%%%%%%%%%%%%%%%%%%%%%%%%%%%%%%%%%%
\begin{frame}[fragile]{Model Variables}
\begin{center}
\begin{itemize}
  \item $Y_t$: number of IPD cases observed in month $t$
  \item $\text{pop}_t$: population
  \item $\text{time}_t$: time in months
  \item $\text{vaccine}_t$: binary, indicates if month $t$ is assigned to the pre- ($\text{vaccine}_t=0$) or to the post-intervention period ($\text{vaccine}_t=1$) \\
  $\Rightarrow$ $\text{vaccine}_t=1$ for $t>48$ and zero otherwise
  \item $\text{time\_after\_vaccine}_t$: number of months elapsed in the post-intervention period \\
  $\Rightarrow$ $\text{time\_after\_vaccine}_t=\text{time}_t-48$ for $t>48$ and zero otherwise
  \item $\sin$, $\cos$, $\sin_{\text{aftervaccine}}$ and $\cos_{\text{aftervaccine}}$: seasonality in pre- and post-intervetion period
\end{itemize}
\end{center}
\end{frame}
%%%%%%%%%%%%%%%%%%%%%%%%%%%%%%%%%%%%%%%%%%%%%%

%\section{Results}
%%%%%%%%%%%%%%%%%%%%%%%%%%%%%%%%%%%%%%%%%%%%%%
\begin{frame}[fragile]{Results: Intervention outcome among $\ge$50 years old}
\begin{center}
\begin{figure}
  \centering
  \caption{Observed incidence/100,000 population, model fit with 95\% CI and secular trend of the segmented regression model including 2012 as transition period. \textit{PCV10 ST-IPD} among $\ge$50 years old, Austria, 2009 -- 2016/05.}
  \includegraphics[width=0.7\linewidth]{age50-120_STPCV10.png}
\end{figure}

\end{center}
\end{frame}
%%%%%%%%%%%%%%%%%%%%%%%%%%%%%%%%%%%%%%%%%%%%%%

%%%%%%%%%%%%%%%%%%%%%%%%%%%%%%%%%%%%%%%%%%%%%%
\begin{frame}[fragile]{Results: Control outcome among $\ge$50 years old}
\begin{center}
\begin{figure}
  \centering
  \caption{Observed incidence/100,000 population, model fit with 95\% CI and secular trend of the segmented regression model including 2012 as transition period. \textit{Non PPV23 ST-IPD} among $\ge$50 years old, Austria, 2009 -- 2016/05.}
  \includegraphics[width=0.7\linewidth]{age50-120_STnonPCV23.png}
\end{figure}

\end{center}
\end{frame}
%%%%%%%%%%%%%%%%%%%%%%%%%%%%%%%%%%%%%%%%%%%%%%

%%%%%%%%%%%%%%%%%%%%%%%%%%%%%%%%%%%%%%%%%%%%%%
\begin{frame}[fragile]{Results: Control outcome among 5--49 years old}
\begin{center}
\begin{figure}
  \centering
  \caption{Observed incidence/100,000 population, model fit with 95\% CI and secular trend of the segmented regression model including 2012 as transition period. \textit{PCV10 ST-IPD} among 5--49 years old, Austria, 2009 -- 2016/05.}
  \includegraphics[width=0.7\linewidth]{age5-49_STPCV10.png}
\end{figure}

\end{center}
\end{frame}
%%%%%%%%%%%%%%%%%%%%%%%%%%%%%%%%%%%%%%%%%%%%%%

%%%%%%%%%%%%%%%%%%%%%%%%%%%%%%%%%%%%%%%%%%%%%%
\begin{frame}[fragile]{Summary}
After the intervention:
\begin{center}
\begin{itemize}
  \item \ge$50 years old
  \begin{itemize}
    \item decline in PCV10 serotypes (intervention outcome)
    \item no secular change in other serotype groups (control outcomes)
  \end{itemize}
  \item 5--49 years old
  \begin{itemize}
    \item no changes in any serotype group (control outcomes)
  \end{itemize}
\end{itemize}
\vspace{1cm}
\begin{block}{Conclusion}
  Indirect effect (herd immunity) among elders after intervention in <5 years old!
\end{block}
\end{center}
\end{frame}
%%%%%%%%%%%%%%%%%%%%%%%%%%%%%%%%%%%%%%%%%%%%%%

%%%%%%%%%%%%%%%%%%%%%%%%%%%%%%%%%%%%%%%%%%%%%%
\begin{frame}[fragile]{<5 years old}
\begin{center}
\begin{itemize}
  \item Low number of cases, especially in subgroups (e.g. PCV10 ST-IPD)
  \begin{itemize}
    \item [$\Rightarrow$] segmented regression not applicable
  \end{itemize}
  \item \glqq{}Molecular\grqq{} laboratory method (PCR) introduced in 2013
    \item Used for case finding and serotype determination
  \begin{itemize}
    \item [$\Rightarrow$] possibly more cases found and more serotypes determined
    \item [$\Rightarrow$] comparison of pre- and post-intervention period not applicable
    \item [$\Rightarrow$] case counts are still low
  \end{itemize}
  \item Use Zero-inflated and Hurdle models
\end{itemize}
\end{center}
\end{frame}
%%%%%%%%%%%%%%%%%%%%%%%%%%%%%%%%%%%%%%%%%%%%%%

%%%%%%%%%%%%%%%%%%%%%%%%%%%%%%%%%%%%%%%%%%%%%%
\begin{frame}[fragile]{Method: Zero-inflated vs Hurdle}
\begin{center}
\begin{itemize}
  \item Developed to model count data with high occurence of zero counts
  \item Usable for Poisson and negative binomial distributed observations
  \item Zeros are interpreted as
  \begin{itemize}
    \item structural zeros - due to some special structure of the data
    \item sampling zeros - from sampling
  \end{itemize}
  \item Zero-inflated: structural and sampling zeros
  \item Hurdle: structural zeros only
%  \item Can be considered as finite mixture models (Cameron and Trivedi \cite{cameron1998})
\end{itemize}
\end{center}
\end{frame}
%%%%%%%%%%%%%%%%%%%%%%%%%%%%%%%%%%%%%%%%%%%%%%

%%%%%%%%%%%%%%%%%%%%%%%%%%%%%%%%%%%%%%%%%%%%%%
\begin{frame}[fragile]{Model: Poisson version of zero-inflated and hurdle model}
\begin{center}
\begin{block}{Zero-inflated model}
\begin{equation*}
P(Y_i = y_i) = \begin{cases}
    p_i + \left(1 - p_i\right) e^{-\mu_i},                     & \quad y_i = 0\\
    \left(1 - p_i\right) \frac{e^{-\mu_i}\mu_i^{y_i}}{y_i!},   & \quad y_i > 0\\
  \end{cases}
\end{equation*}
\end{block}

\begin{block}{Hurdle model}
\begin{equation*}
P(Y_i = y_i) = \begin{cases}
    p_i,                                                                               & \quad y_i = 0\\
    \left(1 - p_i\right) \frac{e^{-\mu_i}\mu_i^{y_i}}{\left(1-e^{-\mu_i}\right)y_i!},  & \quad y_i > 0\\
  \end{cases}
\end{equation*}
\end{block}
\begin{itemize}
  \item $p_i$: probability of being excess zero - often modelled using logistic regression
\end{itemize}
\end{center}
\end{frame}
%%%%%%%%%%%%%%%%%%%%%%%%%%%%%%%%%%%%%%%%%%%%%%

%%%%%%%%%%%%%%%%%%%%%%%%%%%%%%%%%%%%%%%%%%%%%%
\begin{frame}[fragile]{Results using different models}
\begin{center}
% latex table generated in R 3.2.4 by xtable 1.8-2 package
% Fri Jun 02 13:46:43 2017
\begin{table}[ht]
\centering
\caption{Model output (RRs and p-values) of different models using PCV10-ST IPD data including molecular typed cases on <5 years old, Austria, 2012 -- 2016/05} 
\scalebox{0.6}{
\begin{tabular}{p{3cm}lrrrrrrrrrrr}
  \hline
variable & NB & p\textsubscript{NB} & ZINB & p\textsubscript{ZINB} & NBH & p\textsubscript{NBH} & POIS & p\textsubscript{POIS} & ZIP & p\textsubscript{ZIP} & PH & p\textsubscript{PH} \\ 
  \hline
$Intercept$ & 0.26 & 0.002 & 0.27 & <0.001 & 0.27 & 0.026 & 0.26 & <0.001 & 0.27 & <0.001 & 0.27 & 0.026 \\ 
  $linear$ & 0.95 & 0.035 & 0.97 & 0.082 & 0.97 & 0.255 & 0.95 & 0.018 & 0.97 & 0.082 & 0.97 & 0.255 \\ 
  $Intercept_{zero}$ &  &  & 0.00 & 0.620 & 1.26 & 0.732 &  &  & 0.00 & 0.537 & 1.26 & 0.732 \\ 
  $linear_{zero}$ &  &  & 3.05 & 0.672 & 0.95 & 0.091 &  &  & 2.70 & 0.577 & 0.95 & 0.091 \\ 
  $AIC$ & 81.38 &  & 79.44 &  & 88.04 &  & 79.92 &  & 77.45 &  & 86.04 &  \\ 
   \hline
\end{tabular}
}
\end{table}

\end{center}
\end{frame}
%%%%%%%%%%%%%%%%%%%%%%%%%%%%%%%%%%%%%%%%%%%%%%

%%%%%%%%%%%%%%%%%%%%%%%%%%%%%%%%%%%%%%%%%%%%%%
\begin{frame}[fragile]{Conclusion, <5 years old}
\begin{center}
\begin{itemize}
  \item Poisson models fitted data better than their negative binomial counterparts\\
  $\Rightarrow$ no significant overdispersion
  \item Ordinary poisson regression explained \textit{Non PCV10 ST-IPD} and total IPD data best\\
  $\Rightarrow$ no, or less excess zeros compared to \textit{PCV10-ST IPD}
  \item PCV10 serotypes: secular decline (intervention outcome)
  \item Non PCV10 serotypes: no changes (control outcome)
\end{itemize}
\end{center}
\end{frame}
%%%%%%%%%%%%%%%%%%%%%%%%%%%%%%%%%%%%%%%%%%%%%%

%%%%%%%%%%%%%%%%%%%%%%%%%%%%%%%%%%%%%%%%%%%%%%
\begin{frame}[fragile]{}
\begin{center}
\Huge{\textbf{Thank you!}}\\
\Huge{\textbf{Any questions?}}
\end{center}
\end{frame}
%%%%%%%%%%%%%%%%%%%%%%%%%%%%%%%%%%%%%%%%%%%%%%





\nocite{gillings1981, wagner_segmented_2002, penfold2013, rose2006, hu2011, cameron1998}

\renewcommand*{\bibfont}{\scriptsize}
\setbeamertemplate{bibliography item}{\insertbiblabel}
%%%%%%%%%%%%%%%%%%%%%%%%%%%%%%%%%%%%%%%%%%%%%%
\begin{frame}[fragile]{References}
\printbibliography

\end{frame}
%%%%%%%%%%%%%%%%%%%%%%%%%%%%%%%%%%%%%%%%%%%%%%


\end{document}


