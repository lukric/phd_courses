\documentclass{beamer}\usepackage[]{graphicx}\usepackage[]{color}
%% maxwidth is the original width if it is less than linewidth
%% otherwise use linewidth (to make sure the graphics do not exceed the margin)
\makeatletter
\def\maxwidth{ %
  \ifdim\Gin@nat@width>\linewidth
    \linewidth
  \else
    \Gin@nat@width
  \fi
}
\makeatother

\definecolor{fgcolor}{rgb}{0.345, 0.345, 0.345}
\newcommand{\hlnum}[1]{\textcolor[rgb]{0.686,0.059,0.569}{#1}}%
\newcommand{\hlstr}[1]{\textcolor[rgb]{0.192,0.494,0.8}{#1}}%
\newcommand{\hlcom}[1]{\textcolor[rgb]{0.678,0.584,0.686}{\textit{#1}}}%
\newcommand{\hlopt}[1]{\textcolor[rgb]{0,0,0}{#1}}%
\newcommand{\hlstd}[1]{\textcolor[rgb]{0.345,0.345,0.345}{#1}}%
\newcommand{\hlkwa}[1]{\textcolor[rgb]{0.161,0.373,0.58}{\textbf{#1}}}%
\newcommand{\hlkwb}[1]{\textcolor[rgb]{0.69,0.353,0.396}{#1}}%
\newcommand{\hlkwc}[1]{\textcolor[rgb]{0.333,0.667,0.333}{#1}}%
\newcommand{\hlkwd}[1]{\textcolor[rgb]{0.737,0.353,0.396}{\textbf{#1}}}%
\let\hlipl\hlkwb

\usepackage{framed}
\makeatletter
\newenvironment{kframe}{%
 \def\at@end@of@kframe{}%
 \ifinner\ifhmode%
  \def\at@end@of@kframe{\end{minipage}}%
  \begin{minipage}{\columnwidth}%
 \fi\fi%
 \def\FrameCommand##1{\hskip\@totalleftmargin \hskip-\fboxsep
 \colorbox{shadecolor}{##1}\hskip-\fboxsep
     % There is no \\@totalrightmargin, so:
     \hskip-\linewidth \hskip-\@totalleftmargin \hskip\columnwidth}%
 \MakeFramed {\advance\hsize-\width
   \@totalleftmargin\z@ \linewidth\hsize
   \@setminipage}}%
 {\par\unskip\endMakeFramed%
 \at@end@of@kframe}
\makeatother

\definecolor{shadecolor}{rgb}{.97, .97, .97}
\definecolor{messagecolor}{rgb}{0, 0, 0}
\definecolor{warningcolor}{rgb}{1, 0, 1}
\definecolor{errorcolor}{rgb}{1, 0, 0}
\newenvironment{knitrout}{}{} % an empty environment to be redefined in TeX

\usepackage{alltt}
%\setbeamertemplate{background}{\includegraphics[width=\paperwidth,height=\paperheight,keepaspectratio]{ages_background2.png}}

\usepackage[ngerman,english]{babel}
\usepackage[T1]{fontenc}
\usepackage[utf8]{inputenc}

\usepackage{lmodern} %schrift besser lesbar

\usepackage{amsmath, amsthm}
\usepackage{amssymb}

\usetheme{Boadilla}

\definecolor{ages}{RGB}{204,164,0}
%\usecolortheme[named=ages]{structure}

\setbeamercovered{transparent}
\beamertemplatenavigationsymbolsempty
\setbeamertemplate{footline}[frame number]

\let\otp\titlepage
\renewcommand{\titlepage}{\otp\addtocounter{framenumber}{-1}}
\IfFileExists{upquote.sty}{\usepackage{upquote}}{}
\begin{document}
%%%%
\title{4.B Cartesian products}
\subtitle{from Random Walks on infinite Graphs and Groups by W.Woess \newline
\newline
VO Random Processes 2018/2019}
\author{Lukas Richter}
\date[]{27.02.2019} %{25.02.2019}

% Title slide
%%%%%%%%%%%%%%%%%%%%%%%%%%%%%%%%%%%%%%%%%%%%%%
% \frame{
%   \titlepage
% }
\begin{frame}[plain]
 \titlepage
\end{frame}
%%%%%%%%%%%%%%%%%%%%%%%%%%%%%%%%%%%%%%%%%%%%%%

% slide 1
%%%%%%%%%%%%%%%%%%%%%%%%%%%%%%%%%%%%%%%%%%%%%%
\begin{frame}[fragile]{From Chapter 4.A}
%\begin{center}
\begin{definition}
$\mathcal{N}$ (or $(X, P)$) satisfies an $\mathcal{F}$-isoperimetric inequality $IS_{\mathcal{F}}$, if there is a constant $\kappa > 0$ such that
\begin{align*}
 \mathcal{F}(m(A)) \le \kappa a(\partial A)
\end{align*}
for every finite $A \subset X$. \\
If this holds for the SRW then we say that the graph $X$ itself satisfies $IS_{\mathcal{F}}$.
\end{definition}

If $\mathcal{F}(t) = t^{1-\frac{1}{d}} (1 \le d \le \infty)$ then we call it $d$-dimensional isoperimetric inequality, $IS_d$.

%\end{center}
\end{frame}
% %%%%%%%%%%%%%%%%%%%%%%%%%%%%%%%%%%%%%%%%%%%%%%

% slide 2 - Graphs direct and cartesian product
%%%%%%%%%%%%%%%%%%%%%%%%%%%%%%%%%%%%%%%%%%%%%%
\begin{frame}[fragile]{Definitions I}
\begin{itemize}
\item $X_1, X_2$ ... graphs
\item \textbf{Direct product} $X_1 \otimes X_2$: the graph with vertex set $\{x_1x_2, x_i \in X_i\}$
\item Neighbourhood given by: \\
$x_1 x_2 \sim y_1 y_2 \iff x_1\sim y_1 \text{ and } x_2 \sim y_2$
\item \textbf{Cartesian product} $X_1 \times X_2$: same vertex set as above
\item Neighbourhood given by: $x_1 x_2 \sim y_1 y_2 \iff x_1\sim y_1 \text{ and } x_2 = y_2 \text{ or } x_1 = y_1 \text{ and } x_2 \sim y_2$
\end{itemize}
\end{frame}
%%%%%%%%%%%%%%%%%%%%%%%%%%%%%%%%%%%%%%%%%%%%%%

% slide 3 - Networks direct and cartesian product
%%%%%%%%%%%%%%%%%%%%%%%%%%%%%%%%%%%%%%%%%%%%%%
\begin{frame}[fragile]{Definitions II}
Analoguous for networks
\begin{itemize}
\item $\mathcal{N}_1, \mathcal{N}_2$ ... networks with conductance functions $a_1, a_2$
\item \textbf{Direct product} $\mathcal{N}_1 \otimes \mathcal{N}_2$: with conductance function\\
$a(x_1 x_2, y_1 y_2) =  a_1(x_1, y_1)a_2(x_2, y_2)$
\item \textbf{Cartesian product} $\mathcal{N}_1 \times \mathcal{N}_2$: with conductance function\\
$a(x_1 x_2, y_1 y_2) =  a_1(x_1, y_1)\delta_{x_2}(y_2) + a_2(x_2, y_2)\delta_{x_1}(y_1)$
\end{itemize}
\end{frame}
%%%%%%%%%%%%%%%%%%%%%%%%%%%%%%%%%%%%%%%%%%%%%%

% slide 4 - transition matrices direct and cartesian product
%%%%%%%%%%%%%%%%%%%%%%%%%%%%%%%%%%%%%%%%%%%%%%
\begin{frame}[fragile]{Definitions III}
Transition matrices
\begin{itemize}
\item $P_1, P_2$ ... transition matrices over $X_1, X_2$
\item \textbf{Direct product} (tensor product) $P_1 \otimes P_2$:\\
$p(x_1 x_2, y_1 y_2) =  p_1(x_1, y_1)p_2(x_2, y_2)$
\item \textbf{Cartesian product} $P_1 \times P_2$:\\
$P = c P_1 \otimes I_2 + (1 - c) I_1 \otimes P_2$ with $0<c<1$.
\end{itemize}
\end{frame}
%%%%%%%%%%%%%%%%%%%%%%%%%%%%%%%%%%%%%%%%%%%%%%

% slide 5 - Theorem 4.10
%%%%%%%%%%%%%%%%%%%%%%%%%%%%%%%%%%%%%%%%%%%%%%
\begin{frame}[fragile]{Theorem}
\begin{theorem}
Let $\mathcal{N}_i,i=1,2$ be networks with associated invariant measures $m_i: \sup_{X_i} m_i(x)<\infty$. If $\mathcal{N}_1$ satisfies $IS_{d_1}$ and $\mathcal{N}_2$ satisfies $IS_{d_2}$ then $\mathcal{N} = \mathcal{N}_1 \times \mathcal{N}_2$ satisfies $IS_{d_1+d_2}$.
\end{theorem}
\end{frame}
%%%%%%%%%%%%%%%%%%%%%%%%%%%%%%%%%%%%%%%%%%%%%%

% slide X - Remarks
%%%%%%%%%%%%%%%%%%%%%%%%%%%%%%%%%%%%%%%%%%%%%%
\begin{frame}[fragile]{Remarks}
\begin{itemize}
\item Let $\mathcal{N}_1, \mathcal{N}_2$ be networks which satisfy $IS_{d_1}$ and $IS_{d_2}$ and $P_i \ge c_i I_i$ holds elementwise ($c_i > 0, I_i$ identity over $X_i$), then $\mathcal{N}_1 \otimes \mathcal{N}_2$ satisfies $IS_{d_1+d_2}$.
\end{itemize}
\end{frame}
%%%%%%%%%%%%%%%%%%%%%%%%%%%%%%%%%%%%%%%%%%%%%%

% slide X - Corollary
%%%%%%%%%%%%%%%%%%%%%%%%%%%%%%%%%%%%%%%%%%%%%%
\begin{frame}[fragile]{Corollary}
\begin{corollary}
Let $X_1, X_2$ be infinite graphs with bounded geometry. If $X_1$ satisfies $IS_{d_1}$ and $X_2$ satisfies $IS_{d_2}$ then $X_1 \times X_2$ satisfies $IS_{d_1+d_2}$.\\
In every case $X_1 \times X_2$ satisfies $IS_2$.
\end{corollary}
\end{frame}
%%%%%%%%%%%%%%%%%%%%%%%%%%%%%%%%%%%%%%%%%%%%%%

% % slide 3
% %%%%%%%%%%%%%%%%%%%%%%%%%%%%%%%%%%%%%%%%%%%%%%
% \begin{frame}[fragile]{Background}
% \begin{center}
% \begin{itemize}
%   \item 1
%   \item 2
% \end{itemize}
% \end{center}
% \end{frame}
% %%%%%%%%%%%%%%%%%%%%%%%%%%%%%%%%%%%%%%%%%%%%%%

\end{document}


